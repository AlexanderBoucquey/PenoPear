\documentclass{beamer}
\usepackage{graphicx}
\usepackage{subcaption}
\title{Project WIT: Pear}
\author{Mattias Billast, Alexander Boucquey}
\date{\today}

\begin{document}

\begin{frame}
\frametitle{Overzicht project}
\tableofcontents
\end{frame}

\begin{frame}
\frametitle{Werkwijze}
\begin{enumerate}
\item Handgeschreven oplossing;
\item Uitwerken in matlab en verbeteren indien niet realistisch of correct (vergelijken met analytische oplossing);
\item Uitschrijven in c++;
\item Optimaliseren in c++.
\end{enumerate}
\end{frame}

\begin{frame}
\frametitle{Resultaten: Optimal CA}

\begin{columns}
\begin{column}{0.48\textwidth}
\begin{figure}
\includegraphics[width = 1\textwidth]{Optimal_CA_pear_CO2_boven.png}
\caption{CO2 voor Optimal CA.}
\end{figure}
\end{column}
	
\begin{column}{0.48\textwidth}
\begin{figure}
\includegraphics[width = 1\textwidth]{Optimal_CA_pear_O2_boven.png}
\caption{O2 voor Optimal CA.}
\end{figure}
\end{column}
\end{columns}


\end{frame}

\begin{frame}
\frametitle{Resultaten: Disorder inducing}
\begin{columns}
\begin{column}{0.48\textwidth}
\begin{figure}
\includegraphics[width = 1\textwidth]{Disorder_inducing_pear_CO2_boven.png}
\caption{CO2 voor Disorder inducing.}
\end{figure}
\end{column}
	
\begin{column}{0.48\textwidth}
\begin{figure}
\includegraphics[width = 1\textwidth]{Disorder_inducing_pear_O2_boven.png}
\caption{O2 voor Disorder inducing.}
\end{figure}
\end{column}
\end{columns}
	
\end{frame}

\begin{frame}
\frametitle{Resultaten: Precooling}
\begin{columns}
\begin{column}{0.48\textwidth}
\begin{figure}
\includegraphics[width = 1\textwidth]{Precooling_pear_CO2_boven.png}
\caption{CO2 voor Precooling.}
\end{figure}
\end{column}
	
\begin{column}{0.48\textwidth}
\begin{figure}
\includegraphics[width = 1\textwidth]{Precooling_pear_O2_boven.png}
\caption{O2 voor Precooling.}
\end{figure}
\end{column}
\end{columns}
\end{frame}

\begin{frame}
\frametitle{Resultaten: Refrigerator}
\begin{columns}
\begin{column}{0.48\textwidth}
\begin{figure}
\includegraphics[width = 1\textwidth]{Refrigerator_pear_CO2_boven.png}
\caption{CO2 voor Refrigerator.}
\end{figure}
\end{column}
	
\begin{column}{0.48\textwidth}
\begin{figure}
\includegraphics[width = 1\textwidth]{Refrigerator_pear_O2_boven.png}
\caption{O2 voor Refrigerator.}
\end{figure}
\end{column}
\end{columns}
\end{frame}

\begin{frame}
\frametitle{Resultaten: Shelf life en Orchard}
\end{frame}

\begin{frame}
\frametitle{Vergelijk analytische en numerieke oplossing voor cirkel}
\begin{columns}
\begin{column}{0.48\textwidth}
\begin{figure}
\includegraphics[width = 1\textwidth]{Optimal_CA_sphere_O2_analytical.png}
\caption{Vergelijk analytische en numerieke oplossing van O2 voor de Optimal CA.}
\end{figure}
\end{column}
	
\begin{column}{0.48\textwidth}
\begin{figure}
\includegraphics[width = 1\textwidth]{Precooling_sphere_O2_analytical.png}
\caption{Vergelijk analytische en numerieke oplossing van O2 voor de Precooling.}
\end{figure}
\end{column}
\end{columns}
\end{frame}

\begin{frame}
\frametitle{Verloren tijd, onverwachte resultaten en mogelijke verbeteringen}
\begin{itemize}
\item Lang gedacht dat de niet lineaire oplossing in matlab fout was. (Terwijl correct)
\item Een fijner mesh geeft niet persé betere convergentie (wat we wel dachten), dit kan gezien worden bij de analytische en numerieke oplossing.
\item Mogelijke verbeteringen
\begin{itemize}
\item Gebruik maken van het sparse zijn
\item Multigrid methode proberen
\item Zoeken naar een betere preconditioner
\item Opkuisen van cpp file
\item mesh in cpp zelf laten maken (en matlab volledig uitschakelen)
\item ...
\end{itemize}
\end{itemize}
\end{frame}

\end{document}